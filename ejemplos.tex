%Plantilla para tesis del Instituto Tecnológico Superior de Misantla
%Creada por: José Arcángel Salazar Delgado
%Supervisó: Dr. Eddy Sánchez de la Cruz
%Vo.Bo.: MSC. Galdino Martínez Flores.
%Versión: 1.0
\chapter{Ejemplos}

\section{Ejemplo de una lista enumerada}


\begin{enumerate}
	\item uno
	\item dos
	\item tres
\end{enumerate}

Praesent in sapien. Lorem ipsum dolor sit amet, consectetuer adipiscing 
elit. Duis fringilla tristique neque. Sed interdum libero ut metus. 
Pellentesque placerat. Nam rutrum augue a leo. Morbi sed elit sit amet 
ante lobortis sollicitudin.

\section{Ejemplo de insertar una figura}
\begin{figure}[!ht]
	\centering
	\includegraphics[width=4cm, height=4cm]{logo-misantla.jpg}
	\caption{Logo de la institución.}
\end{figure}

\section{Ejemplo de una lista con viñetas}

\begin{itemize}
	\item prueba 
	\item prueba 
	\item prueba
\end{itemize}

\section{Ejemplos de formulas matemáticas}

Sencilla:
\[
m=\frac{y_2-y_1}{x_2-x_1}
\]

Con referencia:\\
\\
Por ejemplo, la ecuación (\ref{eq:pythagoras}):

\begin{equation}\label{eq:pythagoras}
a^2 + b^2 = c^2 .
\end{equation}

Con referencia y comentario:

\begin{equation}\label{eq:ejemplo}
y(x_{i}) = \sin(x_{i})^{2} \ \ \textup{función seno cuadrado}
\end{equation}



Usando tabuladores:
\begin{center}
$t(n)= 
\left \{
\begin{tabular}{c}
$0$ \\
$36$ \\
$5t(n-1)+6t(n-2)+4*3^n$ \\
\end{tabular}
\right \}
$
$\left (
\begin{tabular}{c}
si $n=0$ \\
si $n=1$ \\
en otros casos \\
\end{tabular}
\right )
$
\end{center}

\begin{center}
$t(n)= 
\left \{
\begin{tabular}{c}
$c*n^k$ \\
$a*T(n-b)+c*n^k$ \\
\end{tabular}
\right \}
$
$\left (
\begin{tabular}{c}
si $0\leq n<b$ \\
si $n \geq b $\\
\end{tabular}
\right )
$
\end{center}

\section{Ejemplos de una tabla}

\begin{table}[h!]
	\centering
	\label{tab:table1}
	\begin{tabular}{ccc}
		\toprule
		Some & actual & content\\
		\midrule
		prettifies & the & content\\
		as & well & as\\
		using & the & booktabs package\\
		\bottomrule
	\end{tabular}
	\caption{Ejemplo de tabla.}
\end{table}

\section{Ejemplos de citado}

Ejemplo de citado \cite{Vickrey1961}.

Ejemplo de citado 2 \citeA{Vickrey1961}

Ejemplo de citado 3 \cite{baggio}

Ejemplo de citado 4 \citeA{baggio}

Ejemplo de código

\begin{lstlisting}[language=Python, caption=Python example]
import numpy as np

def incmatrix(genl1,genl2):
m = len(genl1)
n = len(genl2)
M = None #to become the incidence matrix
VT = np.zeros((n*m,1), int)  #dummy variable

#compute the bitwise xor matrix
M1 = bitxormatrix(genl1)
M2 = np.triu(bitxormatrix(genl2),1) 

for i in range(m-1):
	for j in range(i+1, m):
		[r,c] = np.where(M2 == M1[i,j])
		for k in range(len(r)):
			VT[(i)*n + r[k]] = 1;
			VT[(i)*n + c[k]] = 1;
			VT[(j)*n + r[k]] = 1;
			VT[(j)*n + c[k]] = 1;

if M is None:
	M = np.copy(VT)
else:
	M = np.concatenate((M, VT), 1)

VT = np.zeros((n*m,1), int)

return M
\end{lstlisting}