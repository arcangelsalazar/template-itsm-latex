%template para tesis del Instituto Tecnológico Superior de Misantla
%Creado por: José Arcángel Salazar Delgado
\chapter{Planteamiento del problema}

Praesent in sapien. Lorem ipsum dolor sit amet, consectetuer adipiscing 
elit. Duis fringilla tristique neque. Sed interdum libero ut metus. 
Pellentesque placerat. Nam rutrum augue a leo. Morbi sed elit sit amet 
ante lobortis sollicitudin.

\section{Objetivo general y específico}

Ejemplo de una lista enumerada:

\begin{enumerate}
	\item uno
	\item dos
	\item tres
\end{enumerate}

Praesent in sapien. Lorem ipsum dolor sit amet, consectetuer adipiscing 
elit. Duis fringilla tristique neque. Sed interdum libero ut metus. 
Pellentesque placerat. Nam rutrum augue a leo. Morbi sed elit sit amet 
ante lobortis sollicitudin.

Ejemplo de una lista con viñetas

\begin{itemize}
	\item prueba 
	\item prueba 
	\item prueba
\end{itemize}

Ejemplo de insertar una figura:
\begin{figure}[H]
	\caption{Logo de la institución.}
	\centering
	\includegraphics[width=4cm, height=4cm]{logo-misantla.jpg}
\end{figure}

\section{Justificación}


Praesent in sapien. Lorem ipsum dolor sit amet, consectetuer adipiscing 
elit. Duis fringilla tristique neque. Sed interdum libero ut metus. 
Pellentesque placerat. Nam rutrum augue a leo. Morbi sed elit sit amet 
ante lobortis sollicitudin.


Praesent in sapien. Lorem ipsum dolor sit amet, consectetuer adipiscing 
elit. Duis fringilla tristique neque. Sed interdum libero ut metus. 
Pellentesque placerat. Nam rutrum augue a leo. Morbi sed elit sit amet 
ante lobortis sollicitudin.

Ejemplo de una tabla:

\begin{table}[h!]
	\centering
	\caption{Caption for the table.}
	\label{tab:table1}
	\begin{tabular}{ccc}
		\toprule
		Some & actual & content\\
		\midrule
		prettifies & the & content\\
		as & well & as\\
		using & the & booktabs package\\
		\bottomrule
	\end{tabular}
\end{table}

Otro ejemplo de tablas



Ejemplo de citado \cite{Vickrey1961}.

Ejemplo de citado 2 \citeA{Vickrey1961}

Ejemplo de citado 3 \cite{baggio}

Ejemplo de citado 4 \citeA{baggio}

Ejemplo de código

\begin{lstlisting}[language=Python, caption=Python example]
import numpy as np

def incmatrix(genl1,genl2):
m = len(genl1)
n = len(genl2)
M = None #to become the incidence matrix
VT = np.zeros((n*m,1), int)  #dummy variable

#compute the bitwise xor matrix
M1 = bitxormatrix(genl1)
M2 = np.triu(bitxormatrix(genl2),1) 

for i in range(m-1):
for j in range(i+1, m):
[r,c] = np.where(M2 == M1[i,j])
for k in range(len(r)):
VT[(i)*n + r[k]] = 1;
VT[(i)*n + c[k]] = 1;
VT[(j)*n + r[k]] = 1;
VT[(j)*n + c[k]] = 1;

if M is None:
M = np.copy(VT)
else:
M = np.concatenate((M, VT), 1)

VT = np.zeros((n*m,1), int)

return M
\end{lstlisting}

\section{Hipótesis}

